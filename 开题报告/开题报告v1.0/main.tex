\documentclass [16pt] {ctexart}
\usepackage[left=1.5cm, right=1.5cm, top=2cm, bottom=2cm]{geometry}

\title{互联网+时代的共享单车资源配置与调度

开题报告}
\author{小组成员:(排列顺序还要讨论一下)}
\date{}

\begin{document}

\maketitle

\section{选题背景}
\par{近年来,共享单车市场蓬勃发展,便捷的借还服务与高度的灵活性很大程度上解决了市民出行的“最后一公里”问题,为城市生活带来了前所未有的便利。然而,自然调度的不充分性,导致了部分区域的单车资源被闲置,而另一些区域却又产生需求缺口。这使得企业的投放成本大幅上升,用户使用体验严重下降,对共享单车的有效利用产生了负面影响。}
\par{然而,随着“互联网+”技术的出现,单车的配置与调度所能够参考的信息大幅增加,这为单车调度提供了一种全新的可能性。这种技术已经被运用于出租车资源的配置与调度\textsuperscript{\cite{安晓丹2015互联网+}},考虑到出租车与共享单车的相似性,类似的方法应当也适用于共享单车的配置调度。}

\section{研究现状}
\par{目前,针对共享单车投放的研究大多是从投放企业的角度分析问题。\textsuperscript{\cite{于德新2020共享单车调度模型及算法研究}\cite{陈佳惠2021共享单车调度路径优化研究}},还有一些更加入了诸如单车维修一类的实际情况。\textsuperscript{\cite{王涵霄2019考虑维修的共享单车调度优化研究}}这些模型很好地模拟了企业运行过程中可能会遇到的各种问题,因而被广受运营者的青睐。然而,这些模型都 没有很好地建立用户体验和企业成本之间的平衡关系,这导致了最终得出的方案或是成本低廉却忽视了用户体验,或是用户体验良好却成本高昂,产生诸多的不合理之处。}

\section{研究目标}
\par{本课题希望通过对共享单车调度的建模分析,建立衡量共享单车使用有效性的函数,对共享单车资源配置提出一种更为合理,更加有效的解决方案,从而建立一种能在最大化用户体验与便捷度的同时,降低企业在投放、运输、调度过程中成本的单车配置方案,平衡好企业与用户之间的关系,助力解决城市生活的“最后一公里”问题。}

\section{研究计划}
\par{本课题将着眼于构建评价用户满意度与投放合理程度的模型。通过供给量与需求量的比率关于时间的函数衡量不同时段下用户的满意度,再将其与建立的成本模型进行比较。以期较好,较准确地表示用户对于共享单车使用的满意度与投放、调度带来的成本,表示出成本与满意度之间的函数关系。}
\par{然后,通过结合“互联网+”的方式,通过网络爬虫程序获取有关数据,对模型中的参数进行求解,从而获得一种较好的共享单车配置与调度方式。}

\bibliographystyle{IEEEtran}
\bibliography{reference.bib}

\end{document}
